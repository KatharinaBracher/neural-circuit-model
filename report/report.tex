\documentclass[9pt]{article}
\usepackage[onehalfspacing]{setspace}
\usepackage{lipsum}
\usepackage{graphicx}

\usepackage[
	backend=biber,
    style=authoryear,
    %sorting=ynt,
    bibwarn=true,
    bibencoding=utf8,
    sortlocale=de_DE,
    maxbibnames=99,
    maxcitenames=1]{biblatex}
    
\DefineBibliographyStrings{english}{
   andothers = {{et\,al\adddot}},}
   
\addbibresource{thesis.bib}




\begin{document}

\AtBeginEnvironment{figure}{\singlespacing}
\AtBeginEnvironment{table}{\singlespacing}

\begin{titlepage}
	\centering
	{\scshape\LARGE Ludwig-Maximilians-Universität München \par}
	{\scshape\large Department Biologie II Computational Neuroscience \par}
	\vspace{0.9cm}
	\includegraphics[width=0.5\textwidth]{../logo/siegel_black.pdf}\par
	\vspace{1.1cm}
	{\scshape\LARGE Report \par}
	\vspace{0.2cm}
	{\huge\bfseries Computational Simulation of Time Perception: Model Implementation and Description \par}
	\vspace{1.1cm}
	{\Large Katharina \textsc{Bracher} \par}
	{Student ID: 11754625 \par}
	\vspace{0.6cm}
	{\Large Supervision: Dr. Kay \textsc{Thurley} \par}
\end{titlepage}


\normalsize
\tableofcontents

\section{Introduction}
speed of neural trajectories
experimentally found that neural activity in anticipation of a delayed response reaches a fixed threshold with rate inversely proportional to delay period \cite{Wang2018}.
flexible control of speed 
can be achieved by a simple model consisting of two units that have reciprocal inhibitory projections and the speed at which the output evolves can be controlled by a shared input
A potential neural mechanisms for speed control.
\section{Model Description}
\subsection{Circuit}
circuit description
u, v, y each representing the average activity of a neural population
parameter: weights, threshold, tau, initial conditions
noise modeled as independent white noise (stochastic synaptic inputs)
fixed points, dynamical regime (depending on parameter, initial cond), towards stable FP ramp like behavior in y, rate of y inversely related to input I 
input - rate
producing time interval 
\subsection{Update Mechanism and Experiment simulation}
Updating I based on feedback to adjust rate in reproduction
stages: measurement, update and reset, reproduction until threshold 
update: delta y-th, weighted
parameter: memory parameter K, reset, initial conditions, 
threshold 
timeouts
\section{Implementation of Model}
\subsection{Modules}
Euler Implementation to Solve Differential Equation
parallel Simulation
experiment simulation
update mechanism 
\subsection{Structure of Code}
parallel simulations, experiment simulation
\section{Results and Outlook}
experiment simulation plot
behavioral plot
parameter search, extending units, neural trajectories

\end{document}